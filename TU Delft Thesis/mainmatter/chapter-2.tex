\chapter{Background}
\label{chapter:Background}

\emph{Description of chapter, Use the present tense when introducing a chapter or section}

\section{Greenhouse Model}
\label{section:greenhouse-model}
\emph{Short Introduction and background of how crop and greenhouse dynamics can be modelled. And a motivation on why the van hentron model was selected}
\subsection{Model Description}
\emph{The description of the van henten model}

\subsection {Model State Equations}
\emph{The model state equations explained and given}

\emph{Also explain how uncertainty will be treated in the system}

\subsection{Optimization Goal}
\label{ssection:optimization-goal}
\emph{The optimization goal of the algorithms. What it is trying to optimize, i.e. the economic benefit}


\section{Reinforcement Learning}
\subsection{Why RL for Greenhouse Control?}
\emph{why use RL for greenhouse control. and some recent and related works on it}

\subsection{The RL problem}
\emph{A description of what RL aims to solve \\ as well as a description of RL algorithms and Q-learning, policy optimization and and actor critic and which RL algorithm was selected}

\subsection{SAC}
\emph{A brief explanation of how SAC works and how it learns}


\section{MPC}
\subsection{Why MPC for Greenhouse Control?}
\emph{Why should MPC be used for greenhouse control}
\subsection {The General MPC problem}
\emph{What MPC aims to achieve and how}

\subsection{Tracking MPC vs EMPC}
\emph{The differences between EMPC and tracking and the difficulty in determining a suitable cost function and/or terminal constraint for EMPC}

\section{RL and MPC in tandem}
\emph{To drive home the similarities and differences between the two and why they should be merged}
\subsection{The Approach}
\subsection{The Optimality}
\subsection{Computational Effort}
\subsection{The Prediction Horizon}
\subsection{The combination}

