\chapter{Introduction}
\label{chapter:introduction, written in the present tense}

The world population is set to increase to a staggering 10 billion people in the year 2050 \cite{blazhevskaGrowingSlowerPace2019}, increasing food demand substantially. Currently, 800 million people are chronically hungry, with 2 billion people suffering from micronutrient deficiencies \cite{faoFutureFoodAgriculture2017}.The situation is compounded by the anticipated rise in food demand, which is expected to increase from 30\% to 62\% between the years 2010 and 2050, resulting in 30\% of the population being at risk of hunger \cite{vandijkMetaanalysisProjectedGlobal2021}. As such, there is a pressing need to enhance food production by at least 70\% \cite{nishatGreenDealGreenhouse2020}. Although large investments have been made to increase food productivity, food losses, waste, and climate change continue to serve as significant constraints \cite{faoFutureFoodAgriculture2017}. To meet these food demands, agriculture space has drastically increased \cite{winklerGlobalLandUse2021}; however, an increase in agriculture land space has led the sector to account for almost 15\% of the world's energy consumption while also accounting for more than 70\% of water consumption \cite{nishatGreenDealGreenhouse2020}. The need for more efficient use of space and resources is clear to increase food demands while limiting space and resource usage. Although greenhouses have been extensively used to combat these problems and have been shown to reduce the environmental burden as compared to typical open-land production \cite{munozComparingEnvironmentalImpacts2008}, they still require about 10 times more energy consumption compared to traditional farming \cite{nishatGreenDealGreenhouse2020}. This increase in energy is owed to the drastic increase in greenhouse operating costs.  Moreover, with the soaring operating energy costs associated with greenhouses and a global trend indicating an increase in gas and electricity prices \cite{alvarezWhatSoaringEnergy2021}, growers are under increasing pressure to adopt more effective growing methods. As a result, agreement policies have been signed to reduce the $CO_2$ emissions of these greenhouses to an acceptable level \cite{breukersPowerDutchGreenhouse}. \\


The imperative shift towards green practices has led to the emergence of smart greenhouses. These innovative structures are designed to enhance crop yield per hectare and improve food quality by utilizing climate-controlled environments \cite{morcegoReinforcementLearningModel2023}. Such smart greenhouses are essential in combating the degrading effects of climate change on crop quality and yield; however, maintaining such an environment requires advanced control methods. These control methods must be able to adjust factors such as temperature, humidity, lighting, and C02 levels to accommodate ideal conditions for crop growth \cite{devopsGreenhouseClimateControl2021}. Growing crops in a controlled environment can ensure the extension of their growing season as well as protection from outside temperature and weather changes. Moreover, these smart greenhouses must address the additional challenges associated with monitoring, fertilization, irrigation, and pest and disease control of plants \cite{sahooSmartGreenhouseBoon2022}, further necessitating advanced controllers. The advent of smart and advanced greenhouses necessitates skilled labor for operation, contributing to a scarcity of qualified personnel \cite{rusnakWhatCurrentState2018}. Coupled with the escalating labour costs, the move to autonomous greenhouses is an attractive idea. \\


Numerous advanced control strategies for greenhouses have been developed to address the previously mentioned challenges with the advancement of technology. Common nowadays is the use of computers for the control of actuators, adjusting conditions based on set points manually specified by the grower \cite{zhangMethodologiesControlStrategies2020}. While such techniques exist, such as automatic greenhouses, the growth of crops still heavily relies on the expertise of the grower. Given the multitude of factors influencing crop growth, the decision space for setting optimal points becomes immensely complex.  Moreover, this control scheme falls behind state-of-the-art technologies, and it is argued that integrated optimal control ensures the best economic results \cite{vanstratenOptimalGreenhouseCultivation2010}.Therefore, to achieve greater autonomy while staying abreast of technology, control strategies such as Reinforcement Learning (RL) and Model Predictive Control (MPC) have been implemented \cite{zhangMethodologiesControlStrategies2020}. Both strategies provide optimal control to pursue the same goal.  Both types of control schemes offer their respective advantages and disadvantages; however, there is a keen similarity between the two, whereby the combination of the two could result in a more effective solution for autonomous greenhouse control.


\section{Recent and Related Developments}
\emph{Recent works and Related Developments, and the similarity between the works and what this thesis does and does not do}
\section{Problem Statement}
\emph{The aim/objective of the thesis, and the questions that will be answered}

\section{Thesis Contribution}
\emph{What will this thesis contribute to the research community?}

\section{Thesis Outline}
\emph{The general outline of the thesis report}

